% Options for packages loaded elsewhere
\PassOptionsToPackage{unicode}{hyperref}
\PassOptionsToPackage{hyphens}{url}
%
\documentclass[
]{article}
\usepackage{lmodern}
\usepackage{amssymb,amsmath}
\usepackage{ifxetex,ifluatex}
\ifnum 0\ifxetex 1\fi\ifluatex 1\fi=0 % if pdftex
  \usepackage[T1]{fontenc}
  \usepackage[utf8]{inputenc}
  \usepackage{textcomp} % provide euro and other symbols
\else % if luatex or xetex
  \usepackage{unicode-math}
  \defaultfontfeatures{Scale=MatchLowercase}
  \defaultfontfeatures[\rmfamily]{Ligatures=TeX,Scale=1}
\fi
% Use upquote if available, for straight quotes in verbatim environments
\IfFileExists{upquote.sty}{\usepackage{upquote}}{}
\IfFileExists{microtype.sty}{% use microtype if available
  \usepackage[]{microtype}
  \UseMicrotypeSet[protrusion]{basicmath} % disable protrusion for tt fonts
}{}
\makeatletter
\@ifundefined{KOMAClassName}{% if non-KOMA class
  \IfFileExists{parskip.sty}{%
    \usepackage{parskip}
  }{% else
    \setlength{\parindent}{0pt}
    \setlength{\parskip}{6pt plus 2pt minus 1pt}}
}{% if KOMA class
  \KOMAoptions{parskip=half}}
\makeatother
\usepackage{xcolor}
\IfFileExists{xurl.sty}{\usepackage{xurl}}{} % add URL line breaks if available
\IfFileExists{bookmark.sty}{\usepackage{bookmark}}{\usepackage{hyperref}}
\hypersetup{
  pdftitle={Obrada podataka},
  hidelinks,
  pdfcreator={LaTeX via pandoc}}
\urlstyle{same} % disable monospaced font for URLs
\usepackage[margin=1in]{geometry}
\usepackage{color}
\usepackage{fancyvrb}
\newcommand{\VerbBar}{|}
\newcommand{\VERB}{\Verb[commandchars=\\\{\}]}
\DefineVerbatimEnvironment{Highlighting}{Verbatim}{commandchars=\\\{\}}
% Add ',fontsize=\small' for more characters per line
\usepackage{framed}
\definecolor{shadecolor}{RGB}{248,248,248}
\newenvironment{Shaded}{\begin{snugshade}}{\end{snugshade}}
\newcommand{\AlertTok}[1]{\textcolor[rgb]{0.94,0.16,0.16}{#1}}
\newcommand{\AnnotationTok}[1]{\textcolor[rgb]{0.56,0.35,0.01}{\textbf{\textit{#1}}}}
\newcommand{\AttributeTok}[1]{\textcolor[rgb]{0.77,0.63,0.00}{#1}}
\newcommand{\BaseNTok}[1]{\textcolor[rgb]{0.00,0.00,0.81}{#1}}
\newcommand{\BuiltInTok}[1]{#1}
\newcommand{\CharTok}[1]{\textcolor[rgb]{0.31,0.60,0.02}{#1}}
\newcommand{\CommentTok}[1]{\textcolor[rgb]{0.56,0.35,0.01}{\textit{#1}}}
\newcommand{\CommentVarTok}[1]{\textcolor[rgb]{0.56,0.35,0.01}{\textbf{\textit{#1}}}}
\newcommand{\ConstantTok}[1]{\textcolor[rgb]{0.00,0.00,0.00}{#1}}
\newcommand{\ControlFlowTok}[1]{\textcolor[rgb]{0.13,0.29,0.53}{\textbf{#1}}}
\newcommand{\DataTypeTok}[1]{\textcolor[rgb]{0.13,0.29,0.53}{#1}}
\newcommand{\DecValTok}[1]{\textcolor[rgb]{0.00,0.00,0.81}{#1}}
\newcommand{\DocumentationTok}[1]{\textcolor[rgb]{0.56,0.35,0.01}{\textbf{\textit{#1}}}}
\newcommand{\ErrorTok}[1]{\textcolor[rgb]{0.64,0.00,0.00}{\textbf{#1}}}
\newcommand{\ExtensionTok}[1]{#1}
\newcommand{\FloatTok}[1]{\textcolor[rgb]{0.00,0.00,0.81}{#1}}
\newcommand{\FunctionTok}[1]{\textcolor[rgb]{0.00,0.00,0.00}{#1}}
\newcommand{\ImportTok}[1]{#1}
\newcommand{\InformationTok}[1]{\textcolor[rgb]{0.56,0.35,0.01}{\textbf{\textit{#1}}}}
\newcommand{\KeywordTok}[1]{\textcolor[rgb]{0.13,0.29,0.53}{\textbf{#1}}}
\newcommand{\NormalTok}[1]{#1}
\newcommand{\OperatorTok}[1]{\textcolor[rgb]{0.81,0.36,0.00}{\textbf{#1}}}
\newcommand{\OtherTok}[1]{\textcolor[rgb]{0.56,0.35,0.01}{#1}}
\newcommand{\PreprocessorTok}[1]{\textcolor[rgb]{0.56,0.35,0.01}{\textit{#1}}}
\newcommand{\RegionMarkerTok}[1]{#1}
\newcommand{\SpecialCharTok}[1]{\textcolor[rgb]{0.00,0.00,0.00}{#1}}
\newcommand{\SpecialStringTok}[1]{\textcolor[rgb]{0.31,0.60,0.02}{#1}}
\newcommand{\StringTok}[1]{\textcolor[rgb]{0.31,0.60,0.02}{#1}}
\newcommand{\VariableTok}[1]{\textcolor[rgb]{0.00,0.00,0.00}{#1}}
\newcommand{\VerbatimStringTok}[1]{\textcolor[rgb]{0.31,0.60,0.02}{#1}}
\newcommand{\WarningTok}[1]{\textcolor[rgb]{0.56,0.35,0.01}{\textbf{\textit{#1}}}}
\usepackage{graphicx,grffile}
\makeatletter
\def\maxwidth{\ifdim\Gin@nat@width>\linewidth\linewidth\else\Gin@nat@width\fi}
\def\maxheight{\ifdim\Gin@nat@height>\textheight\textheight\else\Gin@nat@height\fi}
\makeatother
% Scale images if necessary, so that they will not overflow the page
% margins by default, and it is still possible to overwrite the defaults
% using explicit options in \includegraphics[width, height, ...]{}
\setkeys{Gin}{width=\maxwidth,height=\maxheight,keepaspectratio}
% Set default figure placement to htbp
\makeatletter
\def\fps@figure{htbp}
\makeatother
\setlength{\emergencystretch}{3em} % prevent overfull lines
\providecommand{\tightlist}{%
  \setlength{\itemsep}{0pt}\setlength{\parskip}{0pt}}
\setcounter{secnumdepth}{-\maxdimen} % remove section numbering

\title{Obrada podataka}
\usepackage{etoolbox}
\makeatletter
\providecommand{\subtitle}[1]{% add subtitle to \maketitle
  \apptocmd{\@title}{\par {\large #1 \par}}{}{}
}
\makeatother
\subtitle{Predavanje 6: Preuzimanje podataka sa interneta (Webscraping)}
\author{true}
\date{}

\begin{document}
\maketitle

{
\setcounter{tocdepth}{4}
\tableofcontents
}
\hypertarget{software-podrux161ka}{%
\subsection{Software podrška}\label{software-podrux161ka}}

\hypertarget{vanjski-software}{%
\subsubsection{``Vanjski'' software}\label{vanjski-software}}

Za današnje pedavanje je potrebno instaliriati
\href{https://selectorgadget.com/}{SelectorGadget}. SelectorGadget je
Chrome ekstenzija koja omogućava jednostavno pronalaženje CSS
selektora.(Instalacija je moguća preko
\href{https://chrome.google.com/webstore/detail/selectorgadget/mhjhnkcfbdhnjickkkdbjoemdmbfginb}{linka}.)
SelectorGadget je dostupan isključivo za Chrome. U slučaju da
preferirate Firefox, opcija je
\href{https://addons.mozilla.org/en-US/firefox/addon/scrapemate/}{ScrapeMate}.

\hypertarget{r-paketi}{%
\subsubsection{R paketi}\label{r-paketi}}

\begin{itemize}
\tightlist
\item
  Novi: \textbf{rvest}, \textbf{janitor}
\item
  Korišteni u prethodnim predavanjima: \textbf{tidyverse},
  \textbf{lubridate}, \textbf{hrbrthemes}
\end{itemize}

Prisjetite se da je \textbf{rvest} automatski instaliran sa
\emph{tidyverse} paketom. Ipak, ovo je prigodan način da instalirate i
učitate sve prethodno pobrojane pakete ukoliko to niste već napravili.

\begin{Shaded}
\begin{Highlighting}[]
\CommentTok{## učitaj i instaliraj pakete}
\ControlFlowTok{if}\NormalTok{ (}\OperatorTok{!}\KeywordTok{require}\NormalTok{(}\StringTok{"pacman"}\NormalTok{)) }\KeywordTok{install.packages}\NormalTok{(}\StringTok{"pacman"}\NormalTok{)}
\NormalTok{pacman}\OperatorTok{::}\KeywordTok{p_load}\NormalTok{(tidyverse, rvest, lubridate, janitor, hrbrthemes)}
\CommentTok{## ggplot2 tema (po želji)}
\KeywordTok{theme_set}\NormalTok{(hrbrthemes}\OperatorTok{::}\KeywordTok{theme_ipsum}\NormalTok{())}
\end{Highlighting}
\end{Shaded}

\hypertarget{webscraping-osnove}{%
\subsection{Webscraping osnove}\label{webscraping-osnove}}

Ovo predavanje se odnosi na preuzimanje sadržaja sa web-a na lokalno
računalo. Svi već imamo iskustvo sa pregledom web sadržaja u našem
browser-u (Chrome, Firefox,\ldots) pa razumijemo da taj sadržaj mora
postojati negdje (podatci). Važno je razumjeti da postoje dva osnovna
načina na koja se web sadržaj prikazuje (\emph{engl.render}) u
browser-u:

\begin{enumerate}
\def\labelenumi{\arabic{enumi}.}
\tightlist
\item
  na strani servera (\emph{Server-side})
\item
  na strani klijenta (\emph{Client side})
\end{enumerate}

\href{https://www.codeconquest.com/website/client-side-vs-server-side/}{Pročitajte}
za više detalje (uključijući primjere). Za potrebe ovog predavanja,
glavni su sljedeći elementi:

\hypertarget{server-strana}{%
\subsubsection{1. Server-strana}\label{server-strana}}

\begin{itemize}
\tightlist
\item
  Skripte koje ``grade'' web stranicu se ne izvršavaju na lokalnom
  računalu nego na (host) serveru koji šalje sav HTML kod.

  \begin{itemize}
  \tightlist
  \item
    npr. Wikipedia tablice su već popupunjene sa svim informacijama
    (brojevi, datumi, nazivi\ldots) koje vidimo u browser-u.
  \end{itemize}
\item
  Drugačije rečeno, sve informacije koje vidimo u našem browser-u su već
  procesuirane od strane (host) servera.
\item
  ``Zamislite'' kao da su informacije ugrađene u HTML web stranice.
\item
  \textbf{Izazov za Webscraping:} Pronaći odgovarajuće CSS (ili Xpath)
  ``selektore''. Snalaženje u dinamičkim web stranicama (npr. ``Next
  page'' i ``Show More'' tabovi).
\item
  \textbf{Ključni koncepti:} CSS, Xpath, HTML
\end{itemize}

\hypertarget{client-strana}{%
\subsubsection{2. Client-strana}\label{client-strana}}

\begin{itemize}
\tightlist
\item
  Web stranica sadržava prazni HTML ili CSS okvi.

  \begin{itemize}
  \tightlist
  \item
    Npr. Moguće je da se stranica sastoji od praznog predloška tablice
    bez ikakvih vrijednosti.
  \end{itemize}
\item
  Kada posjetimo URL takve web stranice, naš browser šalje zahtijev
  (\emph{request}) na host server.
\item
  U slučaju da je sve u redu sa zahtjevom (\emph{valid request}), server
  šalje odgovor (\emph{response}) kao skriptu (\emph{script}), koju naš
  browser izivršava i koristi kako bi popunio HTML predložak sa
  (specifičnim) informacijama koje smo zatražili.
\item
  \textbf{Izazov za Webscraping:} Pronaći ``API točke'' može biti
  problematično pošto one nisu uvijek direktno vidljive.
\item
  \textbf{Ključni koncepti:} API, API točke
\end{itemize}

U ovom predavanju ćemo proći kroz glavne razlike između ova dva pristupa
i dati pregled implikacija koje svaki ima za preuzimanje web sadržaja.
Važno je istaknuti da webscraping uključuje ponešto ``detektivskog''
posla. Često će biti potrebno prilagoditi korake s obzirom na podatke
koje želimo preuzeti, a procedure koje funkcioniraju na jednoj stranici
neće nužno funkcionirati i na drugoj (ponekad neće funkcionirati ni na
istoj nakon nekog vremena!). Zbog toga se je moguće reći da webscraping
podjednako uključuje umjetnost i znanost.

Pozitivna strana priče je da server-strana i client-strana dozvoljavaju
preuzimanje web sadržaja. Kao što ćemo vidjeti u ostatku predavanja,
preuzimanje podataka sa web stranice koja funkcionira na client-strani
(API) je često jednostavnije, pogotovo kada se radi o preuzimanju veće
količine podataka (\emph{bulk}). Za webscraping vrijedi općenito
pravilo: \emph{ako vidite podatke u browseru, možete ih i preuzeti}.

\hypertarget{savjet-etiux10dka-i-zakonska-ograniux10denja}{%
\subsubsection{Savjet: Etička i zakonska
ograničenja}\label{savjet-etiux10dka-i-zakonska-ograniux10denja}}

Prethodna rečenica ne uzima u obzir važne etičke i zakonske aspekte
preuzimanja sadržaja sa interneta. Samo zato što možete nešto preuzeti
sa interneta, ne zanči da biste to i trebali učiniti. Vaša je
odgovornost procijeniti da li web stranica ima zakonska ograničenja na
sadržaj koji se tamo nalazi. Alati koje ćemo koristiti u ovom predavanju
su uistinu moćni i mogu prenapregnuti server i izazvati poteškoće u radu
ili pad web stranice. Glavna krilatice kod webscraping je stoga ``budite
pristojni''!

\hypertarget{webscraping-sa-rvest-paketom-server-strana}{%
\subsection{\texorpdfstring{Webscraping sa \textbf{rvest} paketom
(server-strana)}{Webscraping sa rvest paketom (server-strana)}}\label{webscraping-sa-rvest-paketom-server-strana}}

Glavni paket koji se u R koristi za preuzimanje web sadržaja na strani
severa je\textbf{rvest} (\href{https://rvest.tidyverse.org/}{link}). To
je jednostavan ali moćan paket za webscraping inspiriran Python-ovom
\textbf{Beautiful Soup}
(\href{https://www.crummy.com/software/BeautifulSoup/}{link})
platformom, ali uz dodatne tidyverse funkcionalnosti :-). \textbf{rvest}
je osmišljen za rad sa stranicama koje su procesuirane na srani severa i
zbog toga zahtijeva razumijevanje CSS selektora\ldots pa pogledajmo što
je to točno.

\hypertarget{css-i-selectorgadget}{%
\subsubsection{CSS i SelectorGadget}\label{css-i-selectorgadget}}

Za detaljnije informacije o
\href{https://developer.mozilla.org/en-US/docs/Learn/CSS/Introduction_to_CSS/How_CSS_works}{CSS}
(i.e Cascading Style Sheets) i
\href{http://selectorgadget.com/}{SelectorGadget} pročitajte više na
interentu. Ukratko, CSS je jezik koji određuje izled HTML dokumenata
(uključujući i web stranice). To postiže tako što omogućuje browseru
skup pravila za prikaz koja se formiraju na osnovi:

\begin{enumerate}
\def\labelenumi{\arabic{enumi}.}
\tightlist
\item
  \emph{Properties.} CSS svojstva određuju \textbf{kako} će se nešto
  prkazati. To su npr. fontovi, stilovi, boje, širina stranice itd.
\item
  \emph{Selectors.} CSS selektori odrđuju \textbf{što} što će se
  prikazivati. Oni definirajz pravila koja se pripisuju pojedinim
  elementima stranice. Npr Tekstualni elementi definirani kao ``.h1''
  (i.e.~naslovi) su obično veći i naglašeniji nego elementi definirani
  kao ``.h2'' (i.e.~podnaslovi).
\end{enumerate}

Za preuzimanje sadržaja sa web stranice je bitno identificirati CSS
selektore sadržaja koji želimo skinuti jer tako izoliramo djelove
stranice od interesa. Upravo tu dolazi do izražaja korisnost
\emph{SelectorGadget-a}. U ovom predavanj ućemo proći kroz primjer
korištenja \emph{SelectorGadget-a} no preporučljivo je pogledati
\href{https://cran.r-project.org/web/packages/rvest/vignettes/selectorgadget.html}{vignette}
prije nastavka.

\hypertarget{praktiux10dni-primjer-sprint-na-100m-wikipedia}{%
\subsection{Praktični primjer: Sprint na 100m
(Wikipedia)}\label{praktiux10dni-primjer-sprint-na-100m-wikipedia}}

Stavimo sve ovo u praktični kontekst. Želimo preuzeti podatke sa
Wikipedia stranice
\href{http://en.wikipedia.org/wiki/Men\%27s_100_metres_world_record_progression}{\textbf{Men's
100 metres world record progression}}.

Prvo, otvorite ovu stranicu u vašem browser-u. Upoznajte se sa
strukturom stranice: Kakve objekte stranica sadrži? Koliko ima tablica?
Da li tablice imaju iste kolone? Kakvi su rasponi redova i kolona? itd.

Sada kada ste se upoznali sa strukturom stranice, učitajte cijelu
stranicu u R koristeći \texttt{rvest::read\_html()} funkciju

\begin{Shaded}
\begin{Highlighting}[]
\CommentTok{# library(rvest) ## već učitano}
\NormalTok{m100 <-}\StringTok{ }\KeywordTok{read_html}\NormalTok{(}\StringTok{"http://en.wikipedia.org/wiki/Men%27s_100_metres_world_record_progression"}\NormalTok{) }
\NormalTok{m100}
\end{Highlighting}
\end{Shaded}

\begin{verbatim}
## {html_document}
## <html class="client-nojs" lang="en" dir="ltr">
## [1] <head>\n<meta http-equiv="Content-Type" content="text/html; charset=UTF-8 ...
## [2] <body class="mediawiki ltr sitedir-ltr mw-hide-empty-elt ns-0 ns-subject  ...
\end{verbatim}

Kao što vidite, ovo je \href{https://en.wikipedia.org/wiki/XML}{XML}
dokument\footnote{XML je kratica za Extensible Markup Language i jedan
  je od glavnih jezika za formatiranje web stranica.} koji sadrži sve
potrebno za procesuiranje Wikipedia stranice. To je otprilike kao da
promatrate cjelokupni LaTeX ili .pdf dokument (specifikacije, formule,
itd.), a želite preuzeti samo jednu tablicu ili dio poglavlja.

\hypertarget{tablica-1-pre-iaaf-18811912}{%
\subsubsection{Tablica 1: Pre-IAAF
(1881--1912)}\label{tablica-1-pre-iaaf-18811912}}

Pokušajmo izolirati prvu tablicu sa naslovom
\href{https://en.wikipedia.org/wiki/Men\%27s_100_metres_world_record_progression\#Unofficial_progression_before_the_IAAF}{Unofficial
progression before the IAAF}. Kao što je objašnjeno u rvest vignette,
možemo koristiti funkciju \texttt{rvest::html\_nodes()} kako bismo
izolirali i preuzeli ovu tablicu iz ostatka HTML dokumenta kroz
specifikaciju odggovarajućih CSS selektor.Potom je potrebno pretvoriti
objekt u data frame koristeći \texttt{rvest::html\_table()} funkciju.
Preporuča se korištenje \texttt{fill=TRUE} opcije u ovom slučaju, jer će
se u suprotnom javiti problemi sa formatiranjem redova zbog razmaka u
Wiki tablici.

Koristiti ćemo \href{http://selectorgadget.com/}{SelectorGadget} za
identifikaciju CSS selektora. U ovom slučaju je riječ o ``div+
.wikitable :nth-child(1)'', pa pogledajmo kako to funkcionira.

\begin{Shaded}
\begin{Highlighting}[]
\NormalTok{m100 }\OperatorTok
\StringTok{  }\KeywordTok{html_nodes}\NormalTok{(}\StringTok{"div+ .wikitable :nth-child(1)"}\NormalTok{) }\OperatorTok
\StringTok{  }\KeywordTok{html_table}\NormalTok{(}\DataTypeTok{fill=}\OtherTok{TRUE}\NormalTok{) }
\end{Highlighting}
\end{Shaded}

\begin{verbatim}
## Error in html_table.xml_node(X[[i]], ...): html_name(x) == "table" is not TRUE
\end{verbatim}

Nešto nije u redu\ldots!? Dobili smo error. Bez da ulazimo u detalje,
valja naglasiti da je SelectorGadget ponekad neprecizan\ldots.riječ je o
izvrsnom alatu koji uglavnom radi dobro. Ipak, ponekad ono što izgleda
kao dobar selektor (i.e.~naglašeno žuto) nije ono što točno tražimo. Ovo
je prikazano namjerno radi skretanja pažnje na potencijalne probleme
koji se mogu javiti pri korištenu SelectorGadget. Ponovno valja
istaknuti: Webscraping je u jednakoj mjeri umjetnost i znanost!

Na sreću, postoji i precizniji način određivanja točnog selektora,a
odnosi se na korištenje ``inspect web element'' opcije koju ima
\href{https://www.lifewire.com/get-inspect-element-tool-for-browser-756549}{većina
modernih browser-a}. U ovom slučaju koristimo (\textbf{Ctrl+Shift+I},
ili desni klik miša i izaberi ``Inspect''). Potom ćemo proći kroz
\emph{source elemente} dok Chrome ne istakne tablicu koja nas zanima.
Potom opet desni klik miša i izaberite \textbf{Copy -\textgreater{} Copy
selector}. Ovdje je GIF anaimacija opisane procedure:

\includegraphics{../Foto/inspect100m.gif}

Koristeći ovu metodu dobijemo selektor ``\#mw-content-text
\textgreater{} div \textgreater{} table:nth-child(8)''. Pogledajmo da li
će opvaj put sve funkcionirati bez error-a. Ponovno ćemo koristiti
\texttt{rvest::html\_table(fill=TRUE)} funkciju za prebacivanje tablice
u data frame.

\begin{Shaded}
\begin{Highlighting}[]
\NormalTok{m100 }\OperatorTok
\StringTok{  }\KeywordTok{html_nodes}\NormalTok{(}\StringTok{"#mw-content-text > div > table:nth-child(8)"}\NormalTok{) }\OperatorTok
\StringTok{  }\KeywordTok{html_table}\NormalTok{(}\DataTypeTok{fill=}\OtherTok{TRUE}\NormalTok{) }
\end{Highlighting}
\end{Shaded}

\begin{verbatim}
## [[1]]
##    Time               Athlete    Nationality           Location of races
## 1  10.8           Luther Cary  United States               Paris, France
## 2  10.8             Cecil Lee United Kingdom           Brussels, Belgium
## 3  10.8         Étienne De Ré        Belgium           Brussels, Belgium
## 4  10.8          L. Atcherley United Kingdom     Frankfurt/Main, Germany
## 5  10.8          Harry Beaton United Kingdom      Rotterdam, Netherlands
## 6  10.8 Harald Anderson-Arbin         Sweden         Helsingborg, Sweden
## 7  10.8      Isaac Westergren         Sweden               Gävle, Sweden
## 8  10.8                  10.8         Sweden               Gävle, Sweden
## 9  10.8          Frank Jarvis  United States               Paris, France
## 10 10.8      Walter Tewksbury  United States               Paris, France
## 11 10.8            Carl Ljung         Sweden           Stockholm, Sweden
## 12 10.8      Walter Tewksbury  United States Philadelphia, United States
## 13 10.8          André Passat         France            Bordeaux, France
## 14 10.8            Louis Kuhn    Switzerland            Bordeaux, France
## 15 10.8      Harald Gronfeldt        Denmark             Aarhus, Denmark
## 16 10.8            Eric Frick         Sweden           Jönköping, Sweden
## 17 10.6         Knut Lindberg         Sweden          Gothenburg, Sweden
## 18 10.5         Emil Ketterer        Germany          Karlsruhe, Germany
## 19 10.5           Richard Rau        Germany       Braunschweig, Germany
## 20 10.5           Richard Rau        Germany             Munich, Germany
## 21 10.5            Erwin Kern        Germany             Munich, Germany
##                  Date
## 1        July 4, 1891
## 2  September 25, 1892
## 3      August 4, 1893
## 4      April 13, 1895
## 5     August 28, 1895
## 6      August 9, 1896
## 7  September 11, 1898
## 8  September 10, 1899
## 9       July 14, 1900
## 10      July 14, 1900
## 11 September 23, 1900
## 12    October 6, 1900
## 13      June 14, 1903
## 14      June 14, 1903
## 15       July 5, 1903
## 16     August 9, 1903
## 17    August 26, 1906
## 18       July 9, 1911
## 19    August 13, 1911
## 20       May 12, 1912
## 21       May 26, 1912
\end{verbatim}

Sjajno, čini se da sve radi! Sada ćemo sve pripisati novom objektu
\texttt{pre\_iaaf} i provjeritiobjektnu klasu (class).

\begin{Shaded}
\begin{Highlighting}[]
\NormalTok{pre_iaaf <-}
\StringTok{  }\NormalTok{m100 }\OperatorTok
\StringTok{  }\KeywordTok{html_nodes}\NormalTok{(}\StringTok{"#mw-content-text > div > table:nth-child(8)"}\NormalTok{) }\OperatorTok
\StringTok{  }\KeywordTok{html_table}\NormalTok{(}\DataTypeTok{fill=}\OtherTok{TRUE}\NormalTok{) }
\KeywordTok{class}\NormalTok{(pre_iaaf)}
\end{Highlighting}
\end{Shaded}

\begin{verbatim}
## [1] "list"
\end{verbatim}

Izgleda da smo dobili list-u. Pretvorimo taj objekt \emph{stvarno} u
data frame. To je moguće učiniti na više načina. U ovom slučaju ćemo
koristiti \texttt{dplyr::bind\_rows()} funkciju. Riječ je o izvrsnom
načinu za pretvaranje više list-a u jedan data frame.\footnote{Ovu
  funkciju ćemo susresti još nekoliko puta u daljem tijeku kolegija.}

\begin{Shaded}
\begin{Highlighting}[]
\CommentTok{## pretvori list u data_frame}
\CommentTok{# pre_iaaf <- pre_iaaf[[1]] ## također moguće}
\CommentTok{# library(tidyverse) ## A++već učitano}
\NormalTok{pre_iaaf <-}\StringTok{ }
\StringTok{  }\NormalTok{pre_iaaf }\OperatorTok
\StringTok{  }\KeywordTok{bind_rows}\NormalTok{() }\OperatorTok
\StringTok{  }\KeywordTok{as_tibble}\NormalTok{()}
\NormalTok{pre_iaaf}
\end{Highlighting}
\end{Shaded}

\begin{verbatim}
## # A tibble: 21 x 5
##     Time Athlete            Nationality    `Location of races`   Date           
##    <dbl> <chr>              <chr>          <chr>                 <chr>          
##  1  10.8 Luther Cary        United States  Paris, France         July 4, 1891   
##  2  10.8 Cecil Lee          United Kingdom Brussels, Belgium     September 25, ~
##  3  10.8 Étienne De Ré      Belgium        Brussels, Belgium     August 4, 1893 
##  4  10.8 L. Atcherley       United Kingdom Frankfurt/Main, Germ~ April 13, 1895 
##  5  10.8 Harry Beaton       United Kingdom Rotterdam, Netherlan~ August 28, 1895
##  6  10.8 Harald Anderson-A~ Sweden         Helsingborg, Sweden   August 9, 1896 
##  7  10.8 Isaac Westergren   Sweden         Gävle, Sweden         September 11, ~
##  8  10.8 10.8               Sweden         Gävle, Sweden         September 10, ~
##  9  10.8 Frank Jarvis       United States  Paris, France         July 14, 1900  
## 10  10.8 Walter Tewksbury   United States  Paris, France         July 14, 1900  
## # ... with 11 more rows
\end{verbatim}

Sada je potrebno urediti nazive varijabli (kolona)\ldots ovdje koristimo
\texttt{janitor::clean\_names()} funkciju, koja je napravljena
isključivo za tu namjenu. (Q: Na koji drugi način se to može učiniti?)

\begin{Shaded}
\begin{Highlighting}[]
\CommentTok{# library(janitor) ## učitano}
\NormalTok{pre_iaaf <-}
\StringTok{  }\NormalTok{pre_iaaf }\OperatorTok
\StringTok{  }\KeywordTok{clean_names}\NormalTok{()}
\NormalTok{pre_iaaf}
\end{Highlighting}
\end{Shaded}

\begin{verbatim}
## # A tibble: 21 x 5
##     time athlete            nationality    location_of_races     date           
##    <dbl> <chr>              <chr>          <chr>                 <chr>          
##  1  10.8 Luther Cary        United States  Paris, France         July 4, 1891   
##  2  10.8 Cecil Lee          United Kingdom Brussels, Belgium     September 25, ~
##  3  10.8 Étienne De Ré      Belgium        Brussels, Belgium     August 4, 1893 
##  4  10.8 L. Atcherley       United Kingdom Frankfurt/Main, Germ~ April 13, 1895 
##  5  10.8 Harry Beaton       United Kingdom Rotterdam, Netherlan~ August 28, 1895
##  6  10.8 Harald Anderson-A~ Sweden         Helsingborg, Sweden   August 9, 1896 
##  7  10.8 Isaac Westergren   Sweden         Gävle, Sweden         September 11, ~
##  8  10.8 10.8               Sweden         Gävle, Sweden         September 10, ~
##  9  10.8 Frank Jarvis       United States  Paris, France         July 14, 1900  
## 10  10.8 Walter Tewksbury   United States  Paris, France         July 14, 1900  
## # ... with 11 more rows
\end{verbatim}

Primijetimo da postoji još nešto ``nereda'' u zapisima Isaac-a
Westergren-a u Gävle, Sweden. Mogli bismo to popraviti na nekoliko
načina. U ovom slučaju ćemo pokušati pretvoriti ``athlete'' varijablu
numeričkui zamijeniti je sa prethodnom vrijednosti.

\begin{Shaded}
\begin{Highlighting}[]
\NormalTok{pre_iaaf <-}
\StringTok{  }\NormalTok{pre_iaaf }\OperatorTok
\StringTok{  }\KeywordTok{mutate}\NormalTok{(}\DataTypeTok{athlete =} \KeywordTok{ifelse}\NormalTok{(}\KeywordTok{is.na}\NormalTok{(}\KeywordTok{as.numeric}\NormalTok{(athlete)), athlete, }\KeywordTok{lag}\NormalTok{(athlete)))}
\end{Highlighting}
\end{Shaded}

\begin{verbatim}
## Warning: Problem with `mutate()` input `athlete`.
## i NAs introduced by coercion
## i Input `athlete` is `ifelse(is.na(as.numeric(athlete)), athlete, lag(athlete))`.
\end{verbatim}

\begin{verbatim}
## Warning in ifelse(is.na(as.numeric(athlete)), athlete, lag(athlete)): NAs
## introduced by coercion
\end{verbatim}

Na kraju je potrebno urediti ``date'' varijablu tako da R može
prepoznati string vrijednosti kao datum.

\begin{Shaded}
\begin{Highlighting}[]
\CommentTok{# library(lubridate) ## već učitano}
\NormalTok{pre_iaaf <-}
\StringTok{  }\NormalTok{pre_iaaf }\OperatorTok
\StringTok{  }\KeywordTok{mutate}\NormalTok{(}\DataTypeTok{date =} \KeywordTok{mdy}\NormalTok{(date))}
\NormalTok{pre_iaaf}
\end{Highlighting}
\end{Shaded}

\begin{verbatim}
## # A tibble: 21 x 5
##     time athlete               nationality    location_of_races       date      
##    <dbl> <chr>                 <chr>          <chr>                   <date>    
##  1  10.8 Luther Cary           United States  Paris, France           1891-07-04
##  2  10.8 Cecil Lee             United Kingdom Brussels, Belgium       1892-09-25
##  3  10.8 Étienne De Ré         Belgium        Brussels, Belgium       1893-08-04
##  4  10.8 L. Atcherley          United Kingdom Frankfurt/Main, Germany 1895-04-13
##  5  10.8 Harry Beaton          United Kingdom Rotterdam, Netherlands  1895-08-28
##  6  10.8 Harald Anderson-Arbin Sweden         Helsingborg, Sweden     1896-08-09
##  7  10.8 Isaac Westergren      Sweden         Gävle, Sweden           1898-09-11
##  8  10.8 Isaac Westergren      Sweden         Gävle, Sweden           1899-09-10
##  9  10.8 Frank Jarvis          United States  Paris, France           1900-07-14
## 10  10.8 Walter Tewksbury      United States  Paris, France           1900-07-14
## # ... with 11 more rows
\end{verbatim}

Sada imamo čisti data frame i mogli bismo napraviti vizualizaciju
pre-IAAF podataka. To ćemo ipak malo odgoditi dok ne preuzmemo ostatak
tablica sa stranice\ldots{}

\end{document}
